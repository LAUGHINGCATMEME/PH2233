\documentclass[%
 sor,
 jor,
 amsmath,amssymb,
 reprint,
]{revtex4-2}

\usepackage{graphicx}
\usepackage{xcolor}
\usepackage{siunitx}
\usepackage{dcolumn}
\usepackage{bm}
\usepackage{amsmath, amssymb, amsfonts}
\usepackage{placeins}
\usepackage{float}

\begin{document}

\title{Experiment 2\\Thermal Expansion of Metals}

\author{Aumshree P. Shah\\20231059}
\altaffiliation{\color{red}aumshree.pinkalbenshah@students.iiserpune.ac.in}
\date{13 February 2025}

\begin{abstract}
\centering
In this experiment, we try to determine the thermal expansion coefficient of different metals.
\end{abstract}

\maketitle

\section{Theory and Procedure}
\subsection{Apparatus}
\small
\begin{itemize}
    \item Linear Expansion Apparatus
    \item Brass, Aluminium, and Copper rods
    \item Steam generator and electric heater
    \item Thermometers for temperature measurements
\end{itemize}

\subsection{Theory}
When a solid is heated, its length increases. This expansion is approximately linear for small temperature ranges:
\[
L = L_0 (1 + \alpha \Delta T)
\]
where $L$ is the final length, $L_0$ the initial length, $\alpha$ the coefficient of thermal expansion, and $\Delta T$ the temperature change.

\subsection{Procedure}
\begin{enumerate}
    \item Place the brass rod in the apparatus and ensure the thermometer is in contact with it.
    \item Measure the initial position using a spherometer before heating.
    \item  Pass steam through the apparatus to heat the rod.
    \item  Wait until the temperature stabilizes and measure the new length using the spherometer.
    \item Allow the rod to cool and take repeated measurements.
\end{enumerate}
\pagebreak
\section{Observations}

\begin{table}[h]
\centering
\begin{tabular}{|ccccc|}
    \hline
    Material & $T_i$ (\si{\celsius}) & $T_f$ (\si{\celsius}) &  $\Delta L$ ($10^{-5}$ m) & Length (cm) \\
    \hline
    Copper & 23.9 & 66.7 & 75 &  60.2 \\
    Brass & 27.0 & 77.0 & 77 & 59.9 \\
    Steel & 25.0 & 80.0  & 73 & 60.0 \\
    Steel & 28.0 & 82.0 & 65 & 60.0 \\
    Brass & 24.4 & 80.1 & 85 &  60.0\\
    Aluminium & 25.7 & 75.0 & 102 & 60.0\\
    Copper & 23.1 & 83.4 & 73 & 59.9 \\
    \hline
\end{tabular}
\caption{Data taken on 15 Jan 2025}
\vspace{1cm}
\centering
\begin{tabular}{|ccccc|}
    \hline
    Material & $T_i$ (\si{\celsius}) & $T_f$ (\si{\celsius}) &  $\Delta L$ ($10^{-5}$ m) & Length (cm) \\
    \hline
    Brass & 24.8 & 70.0 & 86&  60.0\\
    Brass & 26.6 & 59.9 & 81 & 60.0\\
    \hline
\end{tabular}
\caption{Data taken on 17 Jan 2025, Rod having uneven temperature, mean  temperature is taken for calculations}

\end{table}

\section{Uncertainties and Error Sources}
\subsection{Measurement Uncertainties}
\begin{itemize}
    \item \textbf{Length Measurements:} Uncertainty of $\pm 0.05$ cm, expansion uncertainty of $\pm 5\times 10^{-6}$ m.
    \item \textbf{Temperature Measurements:} Uncertainty of $\pm 0.005$ \si{\kelvin} due to instrument resolution.
\end{itemize}

\subsection{Systematic Errors}
\begin{itemize}
    \item Steam temperature fluctuations affecting heating consistency.
    \item Inconsistent contact between the thermometer and the metal rod.
    \item Delayed response of the spherometer due to manual operation.
\end{itemize}

\section{Calculation and Error Analysis}
\subsection{Error Propagation}
The uncertainty in $\alpha$ is given by:
\[
\sigma_{\alpha} = \alpha \sqrt{\left( \frac{\sigma_{\Delta L}}{\Delta L} \right)^2 + \left( \frac{\sigma_L}{L} \right)^2 + \left( \frac{\sigma_{\Delta T}}{\Delta T} \right)^2}
\]
where $\sigma_{\Delta L}, \sigma_L, \sigma_{\Delta T}$ are the uncertainties in expansion length, initial length, and temperature difference, respectively.

\subsection{Calculation}
From the Python calculations, we get:\footnote{Refer to \cite{github} for calculations}

\begin{table}[h]
\centering
\begin{tabular}{ccc}
\hline
Material & $\alpha$ (\si{1/\degreeCelsius}) \\
\hline
Copper & $(2.94 \pm 0.12) \times 10^{-5}$ \\
Brass & $(2.61 \pm 0.09) \times 10^{-5}$ \\
Steel & $(2.19 \pm 0.07) \times 10^{-5}$ \\
Steel & $(1.99 \pm 0.07) \times 10^{-5}$ \\
Brass & $(2.52 \pm 0.08) \times 10^{-5}$ \\
Aluminium & $(3.42 \pm 0.12) \times 10^{-5}$ \\
Copper & $(2.00 \pm 0.06) \times 10^{-5}$ \\
Brass & $(3.17 \pm 0.12) \times 10^{-5}$ \\
Brass & $(4.10 \pm 0.19) \times 10^{-5}$ \\
\hline
\end{tabular}
\caption{Calculated expansion coefficients}
\end{table}

\section{Result}
The calculated values of $\alpha$ show high precision but large variations from expected values. The inconsistencies suggest experimental errors, leading to unreliable results.

\textbf{Possible causes:}
\begin{itemize}
    \item Temperature inconsistencies across different trials.
    \item Changing steam temperature.
    \item Uneven heating of the rod.
    \item Not having constant temperature across the rod, making it hard to determine the temperature of it.
\end{itemize}



\noindent\fbox{%
    \parbox{\textwidth}{%
Due to these issues, we consider this experiment is inconclusive in determining values of the coefficients.
       }%
}\\

\noindent\rule{\linewidth}{0.4pt}
\vspace{1cm}

\appendix
\section{Theoretical Values}
The expected values of $\alpha$ in \si{\per\celsius} are:\footnote{\cite{alpha}}
\[
\begin{split}
\alpha_{\text{Steel}} &\approx (1.08 - 1.25) \times 10^{-5}\\
\alpha_{\text{Brass}} &\approx (1.8 - 1.9) \times 10^{-5}\\
\alpha_{\text{Aluminium}} &\approx (2.1 - 2.4) \times 10^{-5}\\
\alpha_{\text{Copper}} &\approx 1.78 \times 10^{-5}
\end{split}
\]

\bibliography{aipsamp}
\end{document}

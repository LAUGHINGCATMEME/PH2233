\documentclass[%
 reprint,
 sor,
 jor,
%superscriptaddress,
%groupedaddress,
%unsortedaddress,
%runinaddress,
%frontmatterverbose, 
%preprint,
%preprintnumbers,
%nofootinbib,
%nobibnotes,
%bibnotes,
 amsmath,amssymb,
 aps,
%pra,
%prb,
%rmp,
%prstab,
%prstper,
%floatfix,
]{revtex4-2}


\usepackage{ulem}
\usepackage{graphicx}
\usepackage{xcolor}
\usepackage{siunitx}
\usepackage{dcolumn}
\usepackage{bm}
\usepackage{amsmath, amssymb, amsfonts}
\usepackage{placeins}
\usepackage{float}



\begin{document}

\title{Experiment 7\\Lee's Method}

\author{Aumshree P. Shah\\20231059}
\altaffiliation{\color{red}aumshree.pinkalbenshah@students.iiserpune.ac.in}
\date{\today}

\begin{abstract}
\centering
In this experiment the thermal conductivity of a bad conductor is measured. 
\end{abstract}

\maketitle

\section{Theory and Procedure}
\subsection{Apparatus}
\small
\begin{minipage}{0.48\textwidth}
\begin{itemize}
\item Lee’s Apparatus \item Bad conductor discs \item Two thermometers \item Boiler and Induction 
\end{itemize}
\end{minipage}
\begin{minipage}{0.48\textwidth}
\begin{itemize}
	\item Stop watch \item Weighing balance \item Vernier Caliper \item Screw gauge
\end{itemize}
\end{minipage}
\subsection{Theory}
Fourier’s law of heat conductance gives the rate of transfer of heat between two objects at temperatures ${T_1}$ and ${T_2}$ connected by a conductor with conductivity $k$, uniform cross-sectional area $A$, and length $l$ as 
\[
\frac{\Delta Q}{\Delta t} = k\,A\,l\,\left({T_2}-{T_1}\right).
\]
This equation governs the rate of heat transfer from disc ${D_2}$ to disc ${D_1}$ (the bottom and top discs of Lee’s apparatus, respectively).\\

The instantaneous rate at which a warm body loses heat to its surroundings is given by Newton’s law of cooling,
\[
\frac{dT}{dt} = -b\,(T-T_a),
\]
where $T_a$ is the ambient temperature. This law governs the rate at which disc ${D_1}$ cools in the second half of the experiment.
If $m$ is the mass of disc ${D_1}$ and $s$ is the specific heat of its material, then the rate at which heat is lost by the disc is 
\[
\frac{\Delta Q_1}{\Delta t} = m\,s\,\frac{dT_1}{dt}.
\]\\

In the steady state achieved in the first half of the experiment, the heat supplied by the steam is balanced by the cooling of disc ${D_1}$. Combining the two heat transfer equations gives the heat balance
\begin{equation}
m\,s\,\frac{dT}{dt} = k\,A\,l\,\left({T_2}-{T_1}\right).
\end{equation}
The value of $\frac{dT}{dt}$ for disc ${D_1}$ can be determined from the cooling curve obtained in the second part of the experiment. As an approximation, a single value of $\frac{dT}{dt}$, calculated at ${T_1}$ during the cooling of disc ${D_1}$ from ${T_1} + \SI{10}{\celsius}$ to ${T_1} - \SI{7}{\celsius}$, is used.
From the known value $s=\SI{0.380}{J\,g^{-1}\,K^{-1}}$ for brass, the conductivity $k$ can be determined.
Note that if the two thermometers do not initially show the same reading, the temperature difference ${T_2}-{T_1}$ must be corrected by the quantity ${T'}$ determined at the beginning of the experiment.

\subsection{Procedure}
\begin{enumerate}
    \item Fill the boiler with water to nearly half and heat it to produce steam. In the meantime, weigh the disc ${D_1}$ on which the apparatus rests.
    \item Measure the diameter of specimen disc $d$ with a vernier calliper and its thickness using a screw gauge at several points, and determine the mean thickness.
    \item Clamp the glass specimen between the base disk ${D_2}$ of the steam jacket and the auxiliary brass disc ${D_1}$. Insert the thermometers (either mercury thermometer or thermocouples) in the two brass disks ${D_1}$ and ${D_2}$.
    \item Check if they show the same readings at room temperature. If not, note the difference $T'$.
    \item Connect the boiler outlet with the inlet of the steam chamber by a rubber tube. Continue passing steam until the two brass disks reach a steady temperature. Note down the temperatures $T_1$ and $T_2$ of the two discs.
    \item The second part of the experiment involves the determination of the cooling rate of disc ${D_1}$ alone. Remove the sample disc. Heat the disc ${D_1}$ directly by the steam chamber until its temperature is about $T_1 + \SI{10}{\celsius}$.
    \item Remove the steam chamber and place the insulating disk on it. Record the temperature of the brass disc at half-minute intervals. Continue until the temperature falls to about $T_1 - \SI{7}{\celsius}$.
\end{enumerate}



\subsubsection{Precautions}
\begin{itemize}
	\item Use thermal gloves while working with the instrument.
	\item Make sure all the contacts are proper.
\end{itemize}



\section{Observations}
\noindent Least count of weighing scale:   \uline{1 g}  \\
Least count of thermometer:  \uline{ 0.5 $\si{\celsius}$ }\\
Least count of vernier calliper:  \uline{$10^{-4}$ m } \\
Least count of screw gauge:  \uline{$10^{-5}$ m } \\
\noindent $T'=$  \uline{$0~\si{\celsius}$}\\
\noindent$M_{D_1} = $ \uline{905 g}\\
\noindent $T_a =$ \uline{$23~\si{\celsius}$}


\begin{table}[h]
\centering
\begin{tabular}{|c|c|c|}
    \hline
    Material & Width (d) $(10^{-5} \si{\meter})$  & Diameter (l) (cm) \\
    \hline
    Glass	& 410-376-376-376-376 & 11.80\\
    Ebonite 	& 203-199-197-201-199 & 11.20-11.30\\
    Rubber 	& 331 & 9.88-10.00\\
    \hline
\end{tabular}
\caption{Data taken on 19 Mar 2025, the different observations are separated by '-'.}

\end{table}


\begin{table}[h]
\centering
\begin{tabular}{|c|c|c|}
    \hline
    Material & $T_1 (\si{\celsius})$  &  $T_1 (\si{\celsius})$ \\
    \hline
    Glass	& 86.0 & 95.0\\
    Ebonite 	& 76.0& 94.5\\
    Rubber 	& 84.0& 95.0 \\
    \hline
\end{tabular}
\caption{Data taken on 19 Mar 2025.}
\end{table}
\begin{table}[H]
\label{tab:}
\centering
  \begin{tabular}{|c|c|}
	  \hline
	  $T~\si{\celsius}$   &  $t$ (sec)\\
	  \hline
 91.5   &7 \\
	  90.5 & 8 \\
	  89.5 & 11 \\
	  88.5 & 11 \\
	  87.5 & - \\
	  86.5 & - \\
	  85.5 & 16 \\
	  84.5 & 17 \\
	  83.5 & - \\
	  82.5 & 21 \\
	  81.5 & 21 \\
	  80.5 & 19 \\
	  79.5 & 23 \\
	  78.5 & 22 \\
	  77.5 & 22 \\
	  \hline
  \end{tabular}
  \caption{Data taken on 21 Mar 2025, the rate of cooling for $D_2$ where $T$ is the mean of floor and ceiling of the one degree temperature range.}
\end{table}


\section{Uncertainties and Error Sources}
\subsection{Measurement Uncertainties}
\begin{itemize}
	\item \textbf{Weight Measurements:} All weight measurements have an uncertainty of 0.5 g.
	\item \textbf{Length Measurements:} Vernier caliper has an uncertainty of $5\times 10^{-5}$m while screw gauge measurements having uncertainty of $5\times 10^{-6}$m.
    \item \textbf{Temperature Measurements:} Uncertainty of $\pm 0.25$ \si{\kelvin} due to instrument resolution.
\end{itemize}

\subsection{Systematic Errors}
\begin{itemize}
	\item Improper thermal contact between disk and thermometer. 
	\item Incomplete contacts between disks.
	\item Non ideal insulating material for calculate the rate of cooling of $D_2$
\end{itemize}



\section{Calculation and Error Analysis}
\subsection{Error Propagation}
Using Equation-1 we get:
$$k=\frac{4ms\dot{T}}{\pi d^2 l(T_2-T_1)}$$
From the length, temperature and mass uncertainty, the error to $k$ will travel using the formula for error propagation as \cite{erroranalysis}: 

\begin{widetext}
	\begin{equation}
\delta k = k\sqrt{
	\left( \frac{\delta m}{ m}\right)^2 + 
	\left( \frac{\delta \dot{T}}{ \dot{T}}\right)^2 + 
	\left( \frac{2\delta d}{ d}\right)^2 + 
	\left( \frac{\delta l}{ l}\right)^2 + 
	\left( \frac{\sqrt 2\delta T}{ T_1 - T_2}\right)^2 
}
\end{equation}
\end{widetext}

\subsection{Calculation}
We calculate the value of $k$ of all data points and their uncertainty from the above formula,  we get (Refer to [3] for calculations):

\begin{table}[H]
\centering
\begin{tabular}{ccc}
\hline
Material & $k$ (\si{\watt\square\per\meter\per\kelvin}) \\
\hline
Glass& $(50.4 \pm 0.3)$ \\
Ebonite & $(42.7\pm 0.2) $ \\
Rubber& $(64.4\pm 0.4) $ \\
\hline
\end{tabular}
\caption{Calculated heat transfer coefficients}
\end{table}

\section{Result}
	    The final heat transfer coefficient values $^{[1]}$\\
\begin{table}[H]
\centering
\begin{tabular}{ccc}
\hline
Material & $k$ (\si{\watt\square\per\meter\per\kelvin}) \\
\hline
Glass& $(50.4 \pm 0.3)$ \\
Ebonite & $(42.7\pm 0.2) $ \\
Rubber& $(64.4\pm 0.4) $ \\
\hline
\end{tabular}
\end{table}











\noindent\rule{\linewidth}{0.4pt}
\vspace{3cm}

\appendix
\section{Theoretical Values}
The theoretical values of $k$ are highly different from the ones found, the source of this errors are described in systematic errors.



\end{document}

\documentclass[%
 sor,
 jor,
 amsmath,amssymb,
 reprint,
]{revtex4-2}

\usepackage{graphicx}
\usepackage{xcolor}
\usepackage{siunitx}
\usepackage{dcolumn}
\usepackage{bm}
\usepackage{amsmath, amssymb, amsfonts}
\usepackage{placeins}
\usepackage{float}
\usepackage{ulem}


\begin{document}

\title{Experiment 8\\Michelson's Interferometer}

\author{Aumshree P. Shah\\20231059}
\altaffiliation{\color{red}aumshree.pinkalbenshah@students.iiserpune.ac.in}
\date{\today}

\begin{abstract}
\centering
In this experiment reflective index of glass and air is measured. 
\end{abstract}

\maketitle

\section{Apparatus}
\small
\begin{minipage}{0.48\textwidth}
\begin{itemize}
\item Lee’s Apparatus \item Bad conductor discs \item Two thermometers \item Boiler and Induction 
\end{itemize}
\end{minipage}
\begin{minipage}{0.48\textwidth}
\begin{itemize}
	\item Stop watch \item Weighing balance \item Vernier Caliper \item Screw gauge
\end{itemize}
\end{minipage}
\section{Theory}
\subsection{Wavelength of Laser Beam}







When a solid is heated, its length increases. This expansion is approximately linear for small temperature ranges:
\begin{equation}
L = L_0 (1 + \alpha \Delta T)
\end{equation}
where $L$ is the final length, $L_0$ the initial length, $\alpha$ the coefficient of thermal expansion, and $\Delta T$ the temperature change.
The above formula can simply be written as: $$\Delta L = L \alpha \Delta T $$ For this experiment we use thermal expansion apparatus as shown in the image:\\

\section{Procedure}
\begin{enumerate}
	\item Read the user manual of the setup \textbf{ENTER A MANUAL HERE}. 
\end{enumerate}
\subsection{Precautions}
\begin{itemize}
	\item Ensure to use thermal gloves, this helps reduce the heat flow from our hand to apparatus
	\item After measuring the expansion, sprinkle water or give enough time for the apparatus to cool (we used wet paper towels to cool the apparatus)
	\item Ensure the screws are tight enough. 
	\item Measure the length of the rod after fixing the screws, measuring the length before may change its value when fixing the screws.
	\item Use different rods, when using the same material to reduce any instrumental error from material.
\end{itemize}
\subsubsection{Calibration}
We start by calibrating the apparatus to see weather the apparatus is working or not. For this we shall use copper rods whose thermal expansion coefficient is known; $\alpha_{\text{copper}} = 16.6 \times 10^{-6} \si{\per\celsius}$\footnote{Reference: \cite{alpha}}. We place the rods in the apparatus and measure its length ($L$), change in length ($\Delta L$) and temperature across the rod, which may not be the same so we take multiple values to reduce error. 
\begin{center}
\begin{tabular}{ |c|c|c|c| } 
 \hline
	Material 	& $T_i (\si{\celsius})$ 	& $L (\si{\centi\meter})$) 	& $\Delta L (10^{-5}\si{\meter})$ \\ 
	\hline
	Copper		& 24.0 				& 59.0 				& 75\\
	Copper		& 25.5-24.5-25.5 		& 59.7				& 74\\	
 \hline
\end{tabular}
\end{center}
Now using the known $\alpha_{\text{copper}}$ we calculate the temperature of rods in equilibrium as: 
\begin{center}
\begin{tabular}{ |c|c|c|c|c|c| } 
 \hline
 Material 	& $T_i (\si{\celsius})$ 	& $L (\si{\centi\meter})$) 	& $\Delta L (10^{-5}\si{\meter})$ & $\alpha (10^{-6}\si{\per\celsius})$	& $T_f (\si{\celsius})$ \\ 
	\hline
 Copper		& 24.0 				& 59.0 				& 75 	& 16.6 & 99.6\\
 Copper		& 25.5-24.5-25.5 		& 59.7				& 74 	& 16.6 & 99.3\\	
 \hline
\end{tabular}
\end{center}
From this data and Appendix-B and we conclude that the temperature of rods in equilibrium is $= 99\pm1~\si{\celsius}$; which we shall take as the final temperature to calculate coefficient of thermal expansion of the other materials. 


\section{Observations}

\begin{table}[h]
\centering
\begin{tabular}{|c|ccc|}
    \hline
    Material & $T_i$ (\si{\celsius})  & Length (cm) & $\Delta L$ ($10^{-5}$ m)\\
    \hline
    Copper 	& 24.0     & 59.8 & 75 \\
    Copper 	& 25.5-24.5-25.5 & 59.7 & 74\\
    Aluminium 	& 24.0-23.0-24.7 & 59.9 & 105\\
    Brass 	& 24.1-23.2-24.3 & 59.7 & 85 \\
    Steel 	& 22.1-24.8-20.5 & - & 74 \\
    Aluminium 	& 24.3-23.7-24.3 & 59.8 & 104\\
    Brass 	& 23.7-22.4-24.3 & 60.0 & 85 \\
    Steel 	& 24.6-25.3 & 59.9 & 76 \\
    Brass	& 24.8-25.3 & 60.1 & 86 \\
    Steel 	& 23.3-23.5 & 59.7 & 76 \\ 
    \hline
\end{tabular}
\caption{Data taken on 11 Mar 2025, the variables represents the property as described in the theory. The '-' value is assumed to be 60.0 cm.}

\end{table}
\noindent Least count of scale: \uline{0.1 cm}  \\
Least count of thermometer:\uline{ 0.1 $\si{\celsius}$ }\\
Least count of spherometer:\uline{$10^{-5}$ m } \\


\section{Uncertainties and Error Sources}
\subsection{Measurement Uncertainties}
\begin{itemize}
    \item \textbf{Length Measurements:} Estimated uncertainty of $\pm 0.1$ cm due to not proper method of viewing, expansion uncertainty of $\pm 5\times 10^{-6}$ m.
    \item \textbf{Temperature Measurements:} Uncertainty of $\pm 0.05$ \si{\kelvin} due to instrument resolution.
\end{itemize}



\section{Calculation and Error Analysis}
\subsection{Error Propagation}
From the length and temperature uncertainty, and using Equation-1 uncertainty in $\alpha$, by the basic formula for error propagation$^{[1]}$ will propogate as .:
\[
\sigma_{\alpha} = \alpha \sqrt{\left( \frac{\sigma_{\Delta L}}{\Delta L} \right)^2 + \left( \frac{\sigma_L}{L} \right)^2 + \left( \frac{\sigma_{\Delta T}}{\Delta T} \right)^2}
\]
where $\sigma_{\Delta L}, \sigma_L, \sigma_{\Delta T}$ are the uncertainties in expansion length, initial length, and temperature difference, respectively.

\subsection{Calculation}
We calculate the value of $\alpha$ of all data points and their uncertainity from hte above formul,  we get (Refer to [3] for calculations):

\begin{table}[h]
\centering
\begin{tabular}{ccc}
\hline
Material & $\alpha$ (\si{1/\degreeCelsius}) \\
\hline
Aluminium & $(2.33 \pm 0.02) \times 10^{-5}$ \\
Aluminium & $(2.32\pm 0.02) \times 10^{-5}$ \\
Brass & $(1.90\pm 0.02) \times 10^{-5}$ \\
Brass & $(1.88\pm 0.02) \times 10^{-5}$ \\
Brass & $(1.92\pm 0.02 \times 10^{-5}$ \\
Copper & $(1.67\pm 0.02) \times 10^{-5}$ \\
Copper & $(1.68\pm 0.02) \times 10^{-5}$ \\
Steel & $(1.61\pm 0.02) \times 10^{-5}$ \\
Steel & $(1.71\pm 0.02) \times 10^{-5}$ \\
Steel & $(1.68\pm 0.02) \times 10^{-5}$ \\

\hline


\end{tabular}
\caption{Calculated expansion coefficients}
\end{table}

\section{Result}
	    The final expansion values by weighted average$^{[1]}$ are:\\
\begin{table}[h]
\centering
\renewcommand{\arraystretch}{1.2}
\begin{tabular}{|c|c|c|c|}
\hline
\textbf{Material} & \textbf{$\alpha$ (1/\textdegree C)} & \textbf{Uncertainty (1/\textdegree C)} & \(\chi^2_\nu\) \\
\hline
Aluminium & \(2.328\times10^{-5}\) & \(6.1\times10^{-8}\) & 0.15 \\
Brass     & \(1.90\times10^{-5}\) & \(1.73\times10^{-7}\) & 2.70 \\
Copper    & \(1.674\times10^{-5}\) & \(3.60\times10^{-8}\) & 0.10 \\
Steel     & \(1.67\times10^{-5}\) & \(3.07\times10^{-7}\) & 11.14 \\
\hline
\end{tabular}
\end{table}














\noindent\rule{\linewidth}{0.4pt}

\newpage
\appendix
\section{Theoretical Values}
The expected values of $\alpha$ in \si{\per\celsius} are [4]:
\[
\begin{split}
\alpha_{\text{Steel}} &= (0.99 - 1.73) \times 10^{-5}\\
\alpha_{\text{Brass}} &= (1.8-1.9) \times 10^{-5}\\
\alpha_{\text{Aluminium}} &= (2.1 - 2.4) \times 10^{-5}\\
\alpha_{\text{Copper}} &= (1.6 - 1.67) \times 10^{-5}
\end{split}
\]

\section{Temperature of rod}\label{rodtemp}
The temperature of rod measured with the application of thermal paste is found to be ranging between $98~\si{\celsius}-99~\si{\celsius}$ (measured on 19 Mar 2025)





\end{document}

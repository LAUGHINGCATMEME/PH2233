\documentclass{article}
\usepackage{amsmath,amssymb,booktabs,array,fancyhdr,geometry,hyperref}
\usepackage[utf8]{inputenc}
\usepackage[titles]{tocloft}

\geometry{a4paper,left=4cm,right=4cm,top=4cm,bottom=4cm}
\pagestyle{fancy}
\fancyhf{}
\renewcommand{\headrulewidth}{0.5pt}
\renewcommand{\footrulewidth}{0.5pt}
\lhead{\textit{Experiment 8: Michelson Interferometer}}
\rhead{\thepage}
\lfoot{Aumshree P. Shah}
\rfoot{\href{mailto:aumshree.pinkalbenshah@students.iiserpune.ac.in}{aumshree.pinkalbenshah@students.iiserpune.ac.in}}

\title{
    \vspace{-2cm}
    \textbf{ Michelson Interferometer} \\
    \large\textit{Experiment 8} \\
    \vspace{0.5cm}
    \large Aumshree P. Shah \\
    \large \today
}
\author{}
\date{}

\begin{document}
\maketitle
\thispagestyle{empty}

\tableofcontents
\clearpage

\setcounter{page}{1}

% ==============================================
\section{Wavelength of Laser Beam}
% ==============================================
\subsection{Theory}
The Michelson interferometer splits a laser beam into two paths using a beam splitter. The recombined beams create interference fringes due to the path difference between the mirrors \( M_1 \) and \( M_2 \). The wavelength \( \lambda \) is calculated using:
\[
\lambda = \frac{2d}{N} \Delta
\]
where:
\begin{itemize}
    \item \( d \): Mirror displacement (corrected by calibration constant \( \Delta \))
    \item \( N \): Number of fringes counted
    \item \( \Delta \): Micrometer calibration constant (determined using a reference laser)
\end{itemize}

\subsection{Calibration}
\begin{itemize}
    \item \textbf{Green laser (\( \lambda = 532 \, \text{nm} \)):} 
    \[
    \Delta = \frac{\text{Actual displacement}}{\text{Micrometer reading}} = \frac{N \lambda / 2}{d'} = \frac{20 \times 532 \times 10^{-6} \, \text{mm}}{2 \times 0.22} = 0.02418
    \]
\end{itemize}

\subsection{Data \& Calculations}
\begin{table}[h]
    \centering
    \caption{Wavelength measurements for red laser (\( \Delta = 0.02418 \))}
    \begin{tabular}{cccccc}
        \toprule
        Trial & \( N \) & Micrometer Divisions & \( d' \) (mm) & \( \lambda \) (nm) \\
        \midrule
        1 & 20 & 32 & 0.32 & 774 \\
        2 & 20 & 31 & 0.31 & 747 \\
        3 & 30 & 42 & 0.42 & 678 \\
        4 & 40 & 58 & 0.58 & 698 \\
        5 & 40 & 46 & 0.46 & 556 \\
        6 & 40 & 57 & 0.57 & 690 \\
        7 & 60 & 80 & 0.80 & 645 \\
        8 & 60 & 82 & 0.82 & 663 \\
        \bottomrule
    \end{tabular}
\end{table}
\[
\text{Average } \lambda = \frac{774 + 747 + \cdots + 663}{8} = 681 \, \text{nm}
\]

\subsection{Result}
The wavelength of the red laser is \( \lambda = \boxed{681 \, \text{nm}} \).

% ==============================================
\section{Refractive Index of Glass Slide}
% ==============================================
\subsection{Theory}
Rotating a glass slide (thickness \( t \)) changes the optical path length. The refractive index \( n \) is given by:
\[
n = \frac{(2t - N\lambda)(1 - \cos\theta)}{2t(1 - \cos\theta) - N\lambda}
\]
where:
\begin{itemize}
    \item \( N \): Number of fringe shifts
    \item \( \theta \): Angle of rotation
    \item \( t \): Thickness of the glass slide (assumed \( t = 1.0 \, \text{mm} \))
\end{itemize}

\subsection{Data \& Calculations}
\begin{table}[h]
    \centering
    \caption{Angle measurements and refractive index (\( \lambda = 681 \, \text{nm} \))}
    \begin{tabular}{ccccc}
        \toprule
        Trial & \( N \) & Left (°) & Right (°) & \( n \) \\
        \midrule
        1 & 50 & 14 & 16 & 1.97 \\
        2 & 50 & 14 & 15 & 2.31 \\
        3 & 50 & 15 & 16 & 1.75 \\
        \bottomrule
    \end{tabular}
\end{table}
\[
\text{Average } n = \frac{1.97 + 2.31 + 1.75}{3} = 1.88
\]

\subsection{Result}
The refractive index of the glass slide is \( n = \boxed{1.88} \).

\subsection{Note}
The higher-than-expected value (\( n \approx 1.5 \) for typical glass) suggests potential errors in the assumed thickness \( t \) or angular measurements.

% ==============================================
\section{Refractive Index of Air}
% ==============================================
\subsection{Theory}
Pressure changes (\( \Delta P \)) alter the optical path length in an air cell. The refractive index \( n \) is derived from:
\[
n - 1 = \left( \frac{m_{AP}}{\Delta P} \right) \frac{\lambda}{2d} P_{\text{atm}}
\]
where:
\begin{itemize}
    \item \( m_{AP} \): Number of fringe shifts
    \item \( d \): Length of the air cell (assumed \( d = 100 \, \text{mm} \))
    \item \( P_{\text{atm}} = 760 \, \text{mm Hg} \)
\end{itemize}

\subsection{Data \& Calculations}
\begin{table}[h]
    \centering
    \caption{Pressure and fringe shift data}
    \begin{tabular}{cc}
        \toprule
        \( m_{AP} \) & \( \Delta P \) (mm Hg) \\
        \midrule
        20 & 250 \\
        20 & 240 \\
        20 & 246 \\
        20 & 242 \\
        \bottomrule
    \end{tabular}
\end{table}
\[
\text{Slope } \frac{m_{AP}}{\Delta P} = \frac{20}{244.5} = 0.0818 \, \text{mm Hg}^{-1}
\]
\[
n - 1 = \left( 0.0818 \times \frac{681 \times 10^{-6}}{2 \times 100} \right) \times 760 = 0.000212
\]

\subsection{Result}
The refractive index of air is \( n = \boxed{1.000212} \), consistent with literature values (\( n \approx 1.0003 \)).



\cite{erroranalysis}



\noindent\rule{\linewidth}{0.4pt}
\vspace{3cm}

\bibliographystyle{plain}
\bibliography{bible}

\end{document}





\end{document}

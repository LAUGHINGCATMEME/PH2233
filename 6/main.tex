\documentclass[%
 sor,
%aip,
%twoside,
%groupedaddress,
%jmp,
 jor,
 amsmath,amssymb,
%preprint,%
 reprint,%
%author-year,%
%author-numerical,%
]{revtex4-2}

\usepackage{graphicx}% Include figure files
\usepackage{xcolor}
\usepackage{circuitikz}
\usepackage{siunitx}
\usepackage{dcolumn}% Align table columns on decimal point
\usepackage{bm}% bold math
\usepackage{amsmath, amssymb, amsfonts}
\usepackage{placeins}
\usepackage{float}
\usepackage{gensymb}
\usepackage{array}
\usepackage{siunitx}
\usepackage{enumitem}

\begin{document}

\preprint{IDK what this does}

\title{Experiment 7\\Millikan's Oil Drop}

\author{Aumshree P. Shah\\20231059\color{red}}
\altaffiliation[\color{red}]{aumshree.pinkalbenshah@students.iiserpune.ac.in}
\date{\today}
\vspace{1cm}
\begin{abstract}
\centering
In this experiment, we attempt to measure Planck's constant and test the validity of the inverse square law.
\end{abstract}
\maketitle
\section{Theory and Procedure}
\subsection{Theory}
It was observed as early as 1905 \textbf{ENTER A REFERENCE HERE} that most metals under influence of radiation, emit electrons. This phenomenon was termed as photoelectric emission. The detailed study of it has shown:
\begin{enumerate}
  \item That the emission process depends strongly on frequency of radiation.
  \item For each metal there exists a critical frequency such that light of lower frequency is unable to liberate member of electrons is strictly proportional to the intensity of this radiation.
  \item The emission of electron occurs within a very short time interval after arrival of the radiation and member of electrons is strictly proportional to the intensity of this radiation. 
\end{enumerate}
The experimental facts given above are among the strongest evidence that the electromagnetic field is quantified and the field consists of quanta of energy $E = h \nu$ where $\nu$ is the frequency of the radiation and $h$ is the Planck’s constant. These quanta are called photons. \\

Further it is assumed that electrons are bound inside the metal surface with an energy $e\phi$, where $\phi$ is called work function. It then follows that if the frequency of the light is such that $h\nu > e \phi$ it will be possible to eject photoelectron, while if $h\nu < e \phi$, it would be impossible. In the former case, the excess energy of quantum appears as kinetic energy of the electron, so that \begin{equation}h\nu = \frac 1 2 mv^2 + e\phi \end{equation} which is the famous photoelectrons equation formulated by Einstein in 1905. The energy of emitted photoelectrons can be measured by simple retarding potential techniques as is done in this experiment.
Retarding potential at which the photocurrent stopped, we call it stopping potential ($V_s$) and it's used to measure the kinetic energy of photoelectrons ($E_e$), we have: 
\begin{equation}E_e = \frac 1 2 mv^2 = eV_s \,\,\,\,\,\,\,\,\,\,\,\,\,\,  \text{or} \,\,\,\,\,\,\,\,\,\,\,\,\,\, V_s = \frac h e \nu-\phi \end{equation}
So when we plot the graph of $V_s$ as a function of $\nu$, the slope has a value of $\frac h e $ and it intercepts on x-axis at $V_s = 0$ and the $\nu$ at the intercept with the slope of the graph gives the work function $\phi$.


\subsection{Procedure}
removed stryofom APpensix

 shall make use of multiple filters, refer to appendix A

PRecautions: 
CLeaned lenses, 
not all lences work
the experiment doesnt work erc
WHY THE FUCK IT IS GIVING NEGATIVE CURRECT
\begin{enumerate}
  \item Read the user manual of the setup \textbf{ENTER A model manual HERE}
  \item 
\end{enumerate}

\subsection{Precautions}
\begin{itemize}
  \item Ensure there is no obstruction between the source and the photo-tube \textbf{ENTER A appenex of prev what happened HERE}
  \item clean al filters
  \item take multiple take multiple filters if they dont work as expected \textbf{ENTER A why HERE}
  \item check weather the filters work
  \item make sure the values are making sence \textbf{ENTER A why HERE}
\end{itemize}




\section{Observations}


\begin{table}[ht]
\centering
    \begin{minipage}[b]{0.48\hsize}\centering
\begin{tabular}{|c|c|}
\hline
\textbf{Filter Wavelength (nm)} & \textbf{Voltage (V)} \\
\hline
635 & 		\\
585 &          \\  
540 &          \\
500 &          \\
460 &          \\

\hline
\end{tabular}
\caption{Stopppnig voltage at blag length and stuff}
\label{tab:table1}
    \end{minipage}
\hfill\vline\hfill
    \begin{minipage}[b]{0.48\hsize}\centering
\begin{tabular}{|c|c|}
\hline
\textbf{Filter Wavelength (nm)} & \textbf{Voltage (V)} \\
\hline
635 & 		\\
585 &          \\  
540 &          \\
500 &          \\
460 &          \\

\hline
\end{tabular}
\caption{Data for inverse square law}
\label{tab:table2}
   \end{minipage}
\end{table}





\section{Uncertainties and Sources of Error}
\subsection{Uncertainties from precision of instrument}
\begin{itemize}
    \item Weight Measurements:
    All weight values have an uncertainty of $\Delta m = \pm 0.05$ \si{\gram} due to instrument resolution.    
    \item Temperature Measurements: Temperature values have an uncertainty of $\Delta T = \pm 0.005$ \si{\kelvin} due to instrument resolution. 
\item Mercury Thermometer: The thermometer used to measure $T_s$ has an uncertainty of $\Delta T_m = \pm 0.5$ \si{\kelvin}
\end{itemize}

\subsection{Systematic Errors and Precautions}  
Potential systematic errors included spillage during transfers, residual material retention in containers, unintended material escape from the calorimeter, heat dissipation through radiation (according to Newton's law of cooling), assumptions of instrument calibration accuracy, post-experiment temperature inconsistencies in the calorimeter due to inadequate calibration, and uneven heating.  

Precautionary measures included preheating reused materials to eliminate residual moisture, repeating tests with procedural anomalies to replace outliers, and maintaining calorimeter thermal stability by refilling them with tap water after each run to standardize initial conditions.


\section{Calculation and Error Analysis}



\subsection{Calculation}
For non observed data potms we take sam einital temperatuer. disdcaard twwo non coper data points and take +- 3 grams of errpr of water equivalent for cimplicity. 

\section{Result}
\ref{github}

The reference resistance was determined to be: $R_0 = 2.90 \pm 7.7\% $. (This error accounts for uncertainty in temperature measurements)\\

The Stefan-Boltzmann fit for the tungsten filament showed:
$$
P = (4.00 \pm 0.066) \times 10^{-14}\ \mathrm{T^4}
$$

The log-log analysis revealed a power-law relationship:
$$
\log P = (3.98 \pm 0.06)\log T + (-30.71 \pm 0.48)
$$

\noindent\fbox{%
    \parbox{\textwidth}{%
       The measured slope of $m = 3.98 \pm 0.06$ agrees with Stefan's law prediction of $P \propto T^4$, as the theoretical value of 4 lies within the experimental uncertainty range.
       }%
}\\
\\

\noindent\rule{\linewidth}{0.4pt}
\vspace{2cm}
\appendix
\section{water equivilane}\label{appendix:prevexp}
ONly single measurement of wateer equivilant taken.
\section{Ambient Temperature}
No measure of ambient temperature is taken in the latter experiment, since it had no use in calculations other than the fact that it can be neglected. 


https://en.wikipedia.org/wiki/Photoelectric_effect THE PHOTEVECTRIC current 0egarive makes no fucking sence
refs: https://www.sesinstruments.in/planck-s-constant-experiment-pc-101-5141096.html
http://www.hep.fsu.edu/~wahl/phy4822/expinfo/photel/SVSPhotel.pdf

\bibliography{aipsamp}
\end{document}






\documentclass[%
 jor,
 amsmath,amssymb,
 reprint,
]{revtex4-2}
\usepackage{ulem}
\usepackage{graphicx}
\usepackage{xcolor}
\usepackage{siunitx}
\usepackage{dcolumn}
\usepackage{bm}
\usepackage{placeins}
\usepackage{float}


\begin{document}

\title{Experiment 9\\Brewster's Angle}

\author{Aumshree P. Shah\\20231059}
\altaffiliation{\color{red}aumshree.pinkalbenshah@students.iiserpune.ac.in}
\date{\today}

\begin{abstract}
\centering
In this experiment reflective index of a transparent material is measured using Brewster's angle.
\end{abstract}

\maketitle
\section{Theory and Procedure}
\subsection{Apparatus}
\small
\begin{itemize}
	\item Breadboard\item Laser diode\item Polariser rotator\item Glass slide
\item Rotation stage\item Photodetector\item Detector output unit
\end{itemize}

\subsection{Theory}
A beam of light incident oon a dielectric transparent material can be resolved into parallel(P) and orthogonal(S) components. These components have different reflection coefficients and Brewster discovered that at a particular angle of incidence $\partial_B$ (called Brewster's angle), the reflection coefficient of P-component goes to zero. At this angle direction of reflectied and transmitted beam are orthogonal to each other. \\

By Snell's law, \begin{equation} \tan \partial_B = n
\end{equation}
where $n$ is the refractive index of the material
\subsection{Procedure}
\begin{enumerate}
	\item Read the user manual \textbf{ENTER A REFRNCE 4 HERE}
	\item Mount diode laser to the laser mount.
	\item Switch on the laser and place the polariser rotator & analyser in front of it so as to make the $E$ field parallel to breadboard.
	\item Mount the glass slide on the rotation stage.
	\item Orient the microscope slide to reflect the laser beam back into the laser output aperture.
	\item Rotate the glass slide slowly and note the corresponding degree with intensity of the reflected beam from the glass slide.
	\item The intensity has a  minimum (almost zero) at Brewster's angle $\partial_B$. 
	\item Using Equation-1, calculate the reflective index $n$.
\end{enumerate}
\subsubsection{Precautions}
\begin{itemize}
	\item Make sure the laser output is larger than the photo detector's input area.
\end{itemize}
\section{Observations}

\begin{table}[h]
\centering
\begin{tabular}{|c|ccc|}
    \hline
    & $T_i$ (\si{\celsius})  & Length (cm) & $\Delta L$ ($10^{-5}$ m)\\
    \hline
    Copper 	& 24.0     & 59.8 & 75 \\
    Copper 	& 25.5-24.5-25.5 & 59.7 & 74\\
    Aluminium 	& 24.0-23.0-24.7 & 59.9 & 105\\
    Brass 	& 24.1-23.2-24.3 & 59.7 & 85 \\
    Steel 	& 22.1-24.8-20.5 & - & 74 \\
    Aluminium 	& 24.3-23.7-24.3 & 59.8 & 104\\
    Brass 	& 23.7-22.4-24.3 & 60.0 & 85 \\
    Steel 	& 24.6-25.3 & 59.9 & 76 \\
    Brass	& 24.8-25.3 & 60.1 & 86 \\
    Steel 	& 23.3-23.5 & 59.7 & 76 \\ 
    \hline
\end{tabular}
\caption{Data taken on 11 Mar 2025, the variables represents the property as described in the theory. The '-' value is assumed to be 60.0 cm.}

\end{table}
\noindent Least count of scale: \uline{0.1 cm}  \\
Least count of thermometer:  \uline{ 0.1 $\si{\celsius}$ }\\
Least count of spherometer:  \uline{$10^{-5}$ m } \\


\section{Uncertainties and Error Sources}
\subsection{Measurement Uncertainties}
\begin{itemize}
    \item \textbf{Length Measurements:} Estimated uncertainty of $\pm 0.1$ cm due to not proper method of viewing, expansion uncertainty of $\pm 5\times 10^{-6}$ m.
    \item \textbf{Temperature Measurements:} Uncertainty of $\pm 0.05$ \si{\kelvin} due to instrument resolution.
\end{itemize}



\section{Calculation and Error Analysis}
\subsection{Error Propagation}
From the length and temperature uncertainty, and using Equation-1 uncertainty in $\alpha$, by the basic formula for error propagation$^{[1]}$ will propogate as .:
\[
\sigma_{\alpha} = \alpha \sqrt{\left( \frac{\sigma_{\Delta L}}{\Delta L} \right)^2 + \left( \frac{\sigma_L}{L} \right)^2 + \left( \frac{\sigma_{\Delta T}}{\Delta T} \right)^2}
\]
where $\sigma_{\Delta L}, \sigma_L, \sigma_{\Delta T}$ are the uncertainties in expansion length, initial length, and temperature difference, respectively.

\subsection{Calculation}
We calculate the value of $\alpha$ of all data points and their uncertainity from hte above formul,  we get (Refer to [3] for calculations):

\begin{table}[h]
\centering
\begin{tabular}{ccc}
\hline
Material & $\alpha$ (\si{1/\degreeCelsius}) \\
\hline
Aluminium & $(2.33 \pm 0.02) \times 10^{-5}$ \\
Aluminium & $(2.32\pm 0.02) \times 10^{-5}$ \\
\hline
\end{tabular}
\caption{Calculated expansion coefficients}
\end{table}

\section{Result}
	    The final expansion values by weighted average$^{[1]}$ are:\\
\begin{table}[h]
\centering
\renewcommand{\arraystretch}{1.2}
\begin{tabular}{|c|c|c|c|}
\hline
\textbf{Material} & \textbf{$\alpha$ (1/\textdegree C)} & \textbf{Uncertainty (1/\textdegree C)} & \(\chi^2_\nu\) \\
\hline
Aluminium & \(2.328\times10^{-5}\) & \(6.1\times10^{-8}\) & 0.15 \\
Brass     & \(1.90\times10^{-5}\) & \(1.73\times10^{-7}\) & 2.70 \\
Copper    & \(1.674\times10^{-5}\) & \(3.60\times10^{-8}\) & 0.10 \\
Steel     & \(1.67\times10^{-5}\) & \(3.07\times10^{-7}\) & 11.14 \\
\hline
\end{tabular}
\end{table}









\appendix
\section{Temperature of rod}\label{rodtemp}
The temperature of rod measured with the application of thermal paste is found to be ranging between $98~\si{\celsius}-99~\si{\celsius}$ (measured on 19 Mar 2025)






\begin{thebibliography}{9}
\bibitem{erroranalysis}
Preston, Daryl W. and Dietz, Eric R., \emph{The Art of Experimental Physics}. Available at: \url{http://ilide.info-daryl-w-preston-eric-r-dietz-the-art-of-exp    erimental-physics-wiley}

\bibitem{wiki_thermalexpansion}
Wikipedia, \emph{Brewster's Angle}, 2025. Available at: \url{https://en.wikipedia.org/wiki/Brewster%27s_angle}

\bibitem{github_code}
LAUGHINGCATMEME, \emph{PH2233 - Code Repository}, 2025. Available at: \url{github.com/LAUGHINGCATMEME/PH2233}
\end{thebibliography}



\end{document}

\documentclass[%
 sor,
%aip,
%twoside,
%groupedaddress,
%jmp,
 jor,
 amsmath,amssymb,
%preprint,%
 reprint,%
%author-year,%
%author-numerical,%
]{revtex4-2}
	
\usepackage{graphicx}% Include figure files
\usepackage{xcolor}
\usepackage{circuitikz}
\usepackage{siunitx}
\usepackage{dcolumn}% Align table columns on decimal point
\usepackage{bm}% bold math
\usepackage{amsmath, amssymb, amsfonts}
\usepackage{placeins}
\usepackage{float}
\usepackage{gensymb)
\usepackage{array}
\usepackage{siunitx}
\usepackage{enumitem}


\begin{document}

\preprint{IDK what this does}

\title{Experiment 4\\Specific Heat of Solids}

\author{Aumshree P. Shah\\20231059\color{red}}
\altaffiliation[\color{red}]{aumshree.pinkalbenshah@students.iiserpune.ac.in}
\date{5  Feb 2025}
\vspace{cm}
\begin{abstract}
In this experiment we measuure (add try to measure) the specific hear capacity of solids.
\end{abstract}
\maketitle
\section{Theory and Procedure}
\subsection{Apparatus}

{\small
\begin{itemize}
\begin{minipage}[t]{0.45\textwidth}
    \item Calorimeter
    \item Shots of material(s)
    \item Weighing balance
    \item Steam chamber
\end{minipage}
\hfill
\begin{minipage}[t]{0.45\textwidth}
    \item Steam generator
    \item Electric heater
    \item Thermometers
\end{minipage}
\end{itemize}
}

\subsection{Theory}

The change in temperature $\Delta T$ of a material when supplied with a fixed amount of heat $\Delta Q$ depends
on the type of material and is inversely proportional to the the mass $m$ of the material. The material dependence
is given the quantity called the specific heat $c$ of the substance. The interdependence of these quantities is
summarised by the equation$$\Delta Q= cm \Delta T$$
The amount of heat required to raise the temperature of a unit mass of a substance by 1\si{\celsius} is called the
specific heat of the substance. The SI unit of specific heat is [\si{\joule\per\kilogram\per\kelvin}]. The specific heat of water is among the highest of all substances. Historically the specific heat of water is arbitrarily set as 1\si{\calorie\per\gram\per\celsius}. The equivalent SI value is 4184 \si{\joule\per\kilogram\per\kelvin}. A related quantity is the molar specific heat $c_m$, which is the specific heat for 1  \si{\mole} of a substance. \\

If $c_s$ is the heat capacity of shots, $\Delta T_s \equiv T_f -T_s$ is the change in temperature of shorts and $m_s$ is the mass of shorts then the heat lost by the shots is
$$\Delta Q_s = c_s m_s \Delta T_s$$
Similarly, heat gained by the water is
$$\Delta Q_w = c_w m_ w\Delta T_w$$
with $c_w$, $m_w$, $T_w$ and $\Delta T_w$ have their proper meanings.  \\ Now if there is no heat absorbed by the calorimeter then $\Delta Q_w = \Delta Q_s$, but that is not the case. To make the equation equal we replace $m_w$ by $m_w+m_f$ where $m_f$ is the equivalent water mass of the calorimeter. Hence we get
\begin{equation}
c_s = \frac{c_w(m_w +m_f)\Delta T_w}{m_s \Delta T_s}
\end{equation} \\

\subsection{Procedure}
First we measure the mass of the shots of each material. Then we place shots of material into the steam chamber and close them using a cork with hole through which we put a thermometer to measure the $T_s$. When the temperature reaches equilibrium, we turn the chamber over and quickly pour all the material into calorimeter with measured mass and temperature of water. We stir it using a different thermometer and measure the equilibrium temperature $T_f$. \\
We repeat this same procedure for different materials and then to measure $m_f$ of the calorimeter, we put water of higher temperature $\theta_{i1} \approx 60$\si{\celsius} into room temperature water inside calorimeter and measuring the equilibrium temperature $\theta_f$; using the same principle:
\begin{equation}
m_f = \frac{m_1\Delta \theta_1 - m_2 \Delta \theta_2}{\Delta \theta_2}
\end{equation} \\

\section{Observations}

\begin{table}[ht]
\begin{minipage}[t]{0.48\textwidth}
\centering
\begin{tabular}{|c|c|c|c|c|}
\hline
\textbf{Material} & \textbf{$m_s$(\si{\gram})} & \textbf{ $m_w$(\si{\gram})} & \textbf{$T_w$\si{\celsius}}  & \textbf{$T_f$\si{\celsius}} \\
\hline
Copper & 100.1 & 90.0 & 23.5 & 30.1 \\
Acryllic & 44.5 & 45.0 & 24.9 & 35.0 \\
Copper\footnote{Took the copper out earlier then equilibrium temperature} & 183.5 & 52.0 & 23.0 & 39.1 \\ 
Ceramic & 31.4 & 69.0 & 23.9 & 29.3 \\
\hline
\end{tabular}
\caption{With calorimeter of type-1, data was taken on 23 Jan 2025}
\label{tab:table1}
Data for Type 1:\\
$m_2 = $ \si\{\gram}
$T_2 = $ \si{\celcius} 
$m_1$ \si{\gram}
$T_1$ \si{\celcius}
$T_f$ \si{\celcius}

\end{minipage}
\hfill
\begin{minipage}[t]{0.48\textwidth}
\centering
\begin{tabular}{|c|c|c|c|c|c|}
\hline
\textbf{Material} & \textbf{$m_s$(\si{\gram})} & \textbf{ $m_w$(\si{\gram})} & \textbf{$T_s$\si{\celsius}} & \textbf{$T_w$\si{\celsius}}  & \textbf{$T_f$\si{\celsius}} \\
\hline
Copper & 55.0 & 176.4 & &26.1 & 30.0 \\
Copper\footnote{Spillage while pouring water} & 171.1 & 154.0 & 90--91 &25.1 & 31.2 \\
Acryllic & 52.3 & 144.4 & 87--88 & 25.1 & 29.3 \\
Ceramic & 39.6 & 132.4 & 80 & 24.8 & 28.6 \\
Copper & 180.0 & 180.0 &  & 22.3 & 28.5 \\
\hline
\end{tabular}
\caption{With calorimeter of type-2, data taken on 29 Jan 2025}
\label{tab:table2}
Data for Type 1:\\
$m_2 = $ \si\{\gram}
$T_2 = $ \si{\celcius} 
$m_1$ \si{\gram}
$T_1$ \si{\celcius}
$T_f$ \si{\celcius}
\end{minipage}
\end{table}

\section{Uncertainties and Sources of Error}


The experimental uncertainties originate from the following sources:
\subsection{Uncertainities from precision of instrument}
\begin{itemize}
    \item Weight Measurement:
    All weight values have an uncertainty of $\Delta m = \pm 0.05$ \si{\gram} due to instrument resolution.    
    \item Temperature Measurement: 
    Temperature values have an uncertainty of $\Delta T = \pm 0.005$ \si{\kelvin} due to instrument resolution. 
\item Mercury Thermometer: Thermometer used to measure $T_s$ has an uncertanity of $\Delta T = \pm 0.5$ \si{\kelvin}
\end{itemize}

\subsection{Systematic Errors}
\begin{itemize}
\item Spillage during pouring of water
\item Material being left in container
\item Material going out of calorimeter
\item material radiationg heat as per newtons law of colloing
\item all equipment is gaining proper values assumption
\item unequal temperature of calorimeter after each run if not calibrated
\item  uneven heating 
\subsection*{Precautions}
\item if a material is already used then we preheat it  to remove water
\item if some error occurs during experiment then we repeat to replace the data point.
\item we pour tapp water in calorimeter after each run to make the temperature constant.
\end{itemize}

Explain error analysis of the following code in latex in this way: 
Section of error analysis
Subsection for each parameter and its error propoation a

\section{Calculation and Error Analysis}
Since the voltage is spread across a noticible range we take the mean value of minimum and maximum voltage to reduce the uncertanity of voltage and current. We shall take the \ref{table3} as the data points. Calculating the avrages from \ref{table2} we get:\\

\subsection{Reduced Current Uncertanity}
After taking mean, the new uncertanity in $A$ is determined by $$\Delta A = \Delta A \left( \frac{V_{max} - V_{min}}{\Delta V} \right)$$
\subsection{Error Propagation}
\subsubsection{Resistance Calculation}
The resistance $R = V/I$ has uncertainty given by:
\begin{equation}
    \frac{\Delta R}{R} = \sqrt{\left(\frac{\Delta V}{V}\right)^2 + \left(\frac{\Delta I}{I}\right)^2}
\end{equation}




\subsection{Result}
\ref{github}

The reference resistance was determined to be: $R_0 = 2.90 \pm 7.7\% $. (This error accounts for uncertainty in temperature measurements)\\

The Stefan-Boltzmann fit for the tungsten filament showed:
$$
P = (4.00 \pm 0.066) \times 10^{-14}\ \mathrm{T^4}
$$

The log-log analysis revealed a power-law relationship:
$$
\log P = (3.98 \pm 0.06)\log T + (-30.71 \pm 0.48)
$$

\noindent\fbox{%
    \parbox{\textwidth}{%
       The measured slope of $m = 3.98 \pm 0.06$ agrees with Stefan's law prediction of $P \propto T^4$, as the theoretical value of 4 lies within the experimental uncertainty range.
       }%
}\\
\\

\noindent\rule{\linewidth}{0.4pt}
\vspace{2cm}
\appendix
\section{Equipment Faulty}\label{appendix:prevexp}
This same experiment was performed on 22 January 2025, whose observations didn't align with Stefan's law. Subsequent investigation revealed the apparatus used during that trial had faulty calibration.
\section{Ambient Temperature}
No measure of ambient temperature is taken in the latter experiment, since it had no use in calculations other than the fact that it can be neglected. 

\bibliography{apssamp}
\end{document}


